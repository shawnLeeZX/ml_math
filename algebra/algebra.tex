% vim:tw=72 sw=2 ft=tex
%         File: thoughts_report.tex
% Date Created: 2013 Jun 18
%  Last Change: 2013 Dec 24
%       Author: hhiker
\documentclass[a4paper]{article}
\usepackage{xltxtra}
\usepackage{xcolor}
\usepackage{xeCJK}
\usepackage{minted}
\usepackage{booktabs}
\usepackage{amsmath}
\usepackage{paralist}
\usepackage[colorlinks=true,linkcolor=red]{hyperref}
% \usepackage{hyperref}
\usepackage{varioref}
\usepackage{cleveref}
\newcommand{\head}[1]{\textbf{#1}}
\setCJKmainfont[BoldFont=SimHei,ItalicFont=SimSun]{KaiTi}
\crefname{figure}{图}{图}
\crefname{table}{表}{表}
\crefname{section}{节}{节}

\newtheorem{theorem}{Theorem}[section]
\newtheorem{lemma}[theorem]{Lemma}
\newtheorem{proposition}[theorem]{Proposition}
\newtheorem{corollary}[theorem]{Corollary}
\newtheorem{definition}{Definition}[section]
\newtheorem{remark}{Remark}[section]

\newenvironment{proof}[1][Proof]{\begin{trivlist}
\item[\hskip \labelsep {\bfseries #1}]}{\end{trivlist}}
\newenvironment{example}[1][Example]{\begin{trivlist}
\item[\hskip \labelsep {\bfseries #1}]}{\end{trivlist}}

\newcommand{\qed}{\nobreak \ifvmode \relax \else
      \ifdim\lastskip<1.5em \hskip-\lastskip
      \hskip1.5em plus0em minus0.5em \fi \nobreak
      \vrule height0.75em width0.5em depth0.25em\fi}

\title{Algebra}
\author{Shuai}

\begin{document}
\maketitle
\tableofcontents
\pagebreak

\section{Start of Journey}

	\subsection{Cartesian Product}

	Algebra, and much of math deals with domain and mapping between domains.
	The domain can be natural number, real number, matrix, vector. And the
	mapping belongs to some universal set -- the set of Cartesian product.

	\begin{definition}
		Let $A$ and $B$ be sets. Then the set $A \times B$ of all ordered
		pairs (a, b) where $a \in A, b \in B$ is called the Cartesian product
		of the sets $A$ and $B$.
	\end{definition}

	I first met this concept at the time I leart the Database course. But I
	do not get a intuitive feeling about it until now.

	Actually, the real plane $R^2$ is a natural example of a Cartesian
	product. Cartesian product is like a stage where all algebra players
	must play on such stage.

	\begin{remark}
		The definition and examples are all in two dimesion, however,
		Cartesian product can extend to any dimension.
	\end{remark}

	\subsection{Algebraic Operations}

	With the objects we can manipulate(this part means set theory, which I
	did not talk about), and the stage their mappings play on(Cartesian
	product), we may try to establish some theory upon them.

	\begin{remark}
		There are no rules existing yet. Object(set) are just object. Mappings
		are just mappings.
	\end{remark}

	With the stage we have now, we may begin to return to the math we are
	used to.

	We are used to the concept of operations, such as addition, subtraction.
	They are abtracted from our daily life. To generalize them,
	mathematicians bring forward the idea of binary algebraic operations,
	which is one of the most fundamental in
	mathematics\cite{dixon2011algebra}.

	\begin{definition}
		Let $M$ be a set. The mapping $\theta: M \times M \rightarrow M$ from
		Cartesian square of $M$ to $M$ is called binary(algebraic) operation
		on set $M$. Thus, corresponding to every ordered pair $(a,b)$ of
		elements, where $a,b \in M$, there is a uniquely defined element
		$\theta(a,b) \in M$. The element $\theta(a,b) \in M$ is called
		composition of the elements a and b relative to this
		operation.\cite{dixon2011algebra}
	\end{definition}

	Then starting from binary algebraic operation, mathematicians build the
	algebraic structure bit by bit.

	\begin{remark}
		There is one important note given about notation in the
		book\cite{dixon2011algebra}. It is often rather cumbersome to keep
		referring to the function $\theta$ and using the notation
		$\theta(a,b)$. There are several shorthand symbols that are employed
		and $\theta(a,b)$ is often written using such special notation. For
		example, the operation might be denoted by $\diamond$ and we might
		then write $\theta(a,b) = a \diamond b$. We note that, in general,
		$\theta(a,b)$ will be different from $\theta(b,a)$. However, quite
		often, even the notation $a \diamond b$ is confusing, and most often
		we would rather write the operation $\diamond$ using something more
		familiar. The most familiar binary operators are $+$  and $\cdot$ and
		it is these symbols that are most often useful in writing such
		operations. Thus, instead of writing $a \diamond b$ we may write $a +
		b$ or $ a \cdot b$. It is important to understand that sometimes
		these symbols will have familiar meanings, but not
		always.\cite{dixon2011algebra}
	\end{remark}

	\subsection{Important Properties of Algebraic Structure}

	\subsubsection{Rules}

	\begin{definition}
		\textbf{Commutativity}: A binary operation on a set $M$ is called commutative if $ab = ba$ for
		each pair $a,b$ of elements of $M$.\cite{dixon2011algebra}
	\end{definition}

	\begin{definition}
		\textbf{Associativity}:A binary operation on a set $M$ is called associative if $(ab)c =
		a(bc)$ for each triple $a,b,c$ of elements of
		$M$.\cite{dixon2011algebra}
	\end{definition}

	\subsubsection{Special Elements}

	Besides those two rules, the zero and identity element in one algebraic
	structure are of special status.

	\begin{definition}
		\textbf{Neutral Element}:Let $M$ be a set with binary operation. The
		element $e \in M$ is called a neutral element under this operation if
		$ae = ea = a$ for each element $a$ of the set
		$M$.\cite{dixon2011algebra}
	\end{definition}

	\begin{remark}
		If the operation on $M$ is written multiplicatively, then the term
		\textit{identity element} is usually ued rather than neutral element
		and often $e$ is denoted by $1$ or $1_m$. If we use the additive form,
		then the neutral element is usually called the \textit{zero element}
		and is often denoted by $0_M$, so that the definition of the zero
		element is $ a + 0_M = 0_M + a = a$ for each element $a \in
		M$.\cite{dixon2011algebra}
	\end{remark}

	\subsubsection{Algebraic Properties}

	Properties can also be understanded as restriction put on algebraic
	structure or a abstraction of natural properties of natural algebraic
	structure.

	\begin{definition}
		\textbf{Stable}: Let $M$ be a set with a binary operation. A subset S
		is called stable under this operation if for each pair of elements
		$a,b \in S$ the element ab also belongs to S.\cite{dixon2011algebra}
	\end{definition}

	\begin{definition}
		\textbf{Invertibility}: Let $M$ be a set with binary operation and
		suppose that there is an identity element $e$. The element $x$ is
		called an inverse of the element $a$ if 
		\begin{displaymath}
			ax = xa = e.
		\end{displaymath}
		if $a$ has an inverse then we say that $a$ is invertible.\cite{dixon2011algebra}
	\end{definition}

	\begin{remark}
		Invertibility is just a property. We are used to take it as granted if we
		are so used to the natural operation such as addition or
		multiplication. Some algebraic structure has it, but some do not.
	\end{remark}

	\subsection{Algebraic Structures}

	\begin{definition}
		\textbf{semigroup}: A nonempty set $S$ is called a semigroup if $S$
		has an associative binary operation defined on it. If this operation i
		commutative, we will say that $S$ is a commutative semigroup.\cite{dixon2011algebra}
	\end{definition}

	\begin{definition}
		\textbf{group}: A semigroup $G$ with identity is called a group if
		every element of $G$ is invertible. Thus, a group is a set $G$
		together with a binary algebraic operation $(x,y) \rightarrow xy$
		where $x,y \in G$, such that the following conditions(the group axiom)
		holds:\cite{dixon2011algebra}
		\begin{compactitem}
		\item \textbf{G 1} The operation is associative so that $x(yz) =
			(xy)z$ for all $x,y,z \in G$.
		\item \textbf{G 2} $G$ has an identity element, an element $e$ such
			that $xe = ex = x$ for all $x \in G$; often $1$ or $1_G$ is used in
			place of $e$.
		\item \textbf{G 3} Every element $x \in G$ has an inverse $x^{-1}$
			such that $xx^{-1} = x^{-1}x = e$.
		\end{compactitem}
	\end{definition}

	\begin{definition}
		\textbf{abelian group}: If the group operation is commutative, then
		the group is callled abelian(in honor of the great Norwegian
		mathematician Nielss Henrik Abel(1802 - 1829).\cite{dixon2011algebra}
	\end{definition}

	\subsubsection{Mapping Properties}

	\begin{definition}
		Let $M, S$ be sets with binary operations that we denote by $\ast$ and
		$\diamond$, respectively. Let $ f:M \rightarrow S $ be a mapping. Then
		$f$ is called a homomorphism, if 
		\begin{displaymath}
			f(x\ast y) = f(x) \diamond f(y)
		\end{displaymath}
		for arbitrary elements $x,y \in M$.\cite{dixon2011algebra}
	\end{definition}

	We say that the mapping $f$ respects the operations. An injective
	homomorphism is called \textbf{monomorphism}. A surjective homomorphism is called
	an \textbf{epimorphism} and a bijective homomorphism is called an
	\textbf{isomorphism}.\cite{dixon2011algebra}

	When two structures $M,S$ are isomorphic in this way, there is no
	difference between the structures other than the names we give to the
	elements of the two sets $M$ and $S$ and the names $\ast$ and $\diamond$
	that we give to the names of the operators. Other than this, the
	structures of $M$ and $S$ are identical.\cite{dixon2011algebra}

	If $M$ is a set with binary operation, then the study of $M$ has two
	aspects. The first aspect is concerned with the nature of the elements
	and the structure of $M$, while the second one concerns properties of
	the operation. This enables such a study to be conducted from different
	points of view. We can sdtudy the relationship between the elements and
	the subsets of $M$ and also study individual properties with repsect to
	given operation. Such an approach is feasible for the study of concrete
	sets, such as permutations, transformations of the plane and space,
	symmetries, matrices, and so on. However, we can conduct a study of the
	properties that does not depend on the nature of the elements and which
	is completely defined by the operation. This approach is the key
	approach in algebra and it can be covered by very efficienty, thanks to
	the fundamental notion of isomorphism. Making this more concrete,
	Gottfired Leibniz(1646 - 1716) introduced the general notion of an
	isomorphic relation(which he called a similarity) and pointed out the
	possibility of the identification of isomorphic operations and
	relations. He brought attention to a classical example of isomorphism,
	namely the mapping $x \rightarrow logx$ from the set of all positive
	real numbers with operation of multiplication to the set of all real
	numbers with the operation of addition. A great French mathematician,
	Evariste Galios(1811 - 1832), was also familiar with the idea of
	isomorphism. He understood the corresponding elements of isomorphic sets
	$M$ and $S$ have the same properties with repsect to the given
	operation. This notion in its general form was developed in the middle
	of the nineteenth century. In abstract algebra, we study only such
	properties that are unchanged by isomorphisms.\cite{dixon2011algebra}

\section{Fields}

After becoming familiar with basic algebraic structure and their
properties, I finally reach what I want to understand -- the algebraic
structure, field.

\begin{definition}
	\textbf{Division Ring}: A set $D$ with two binary algebraic
	operations, addition and multiplication, is called a divsion ring if
	it satisfies the following properties:
	\begin{compactitem}
	\item
		the addition is commutative, so 
		\begin{displaymath}
			x + y = y + x
		\end{displaymath}
		for all elements $x,y \in D$;
	\item
		the addition is associative, so
		\begin{displaymath}
			x + (y + z) = (x + y) + z
		\end{displaymath}
		for all elements $x,y,z \in D$;
	\item
		$D$ has a zero element, $0_D$, an element with the property that 
		\begin{displaymath}
			x + 0_D = 0_D + x = x
		\end{displaymath}
		for all elements $x \in D$.
	\item
		each element $x \in D$ has an additive inverse(the opposite or
		negative element), $-x \in D$, an element with the property that
		\begin{displaymath}
			x + (-x) = 0_D;
		\end{displaymath}
	\item
		the distributive laws hold in $D$, so
		\begin{displaymath}
			x(y + z) = xy + xz and (x + y)z = xz + yz
		\end{displaymath}
		for all elements $x,y,z \in D$;
	\item
		the multiplication is associative, so
		\begin{displaymath}
			x(yz) = (xy)z
		\end{displaymath}
		for all elements $x,y,z \in D$;
	\item
		$D$ has a (multiplicative) identity element, $e \neq 0_D$, and
		element with property that
		\begin{displaymath}
			xe = ex = x
		\end{displaymath}
		for each element $x \in D$.
	\item
		each nonzero element $x \in D$ has a multiplicative inverse(the
		reciprocal), $x^{-1} \in D$, and element with property
		\begin{displaymath}
			xx^{-1} = x^{-1}x = e
		\end{displaymath}
	\end{compactitem}\cite{dixon2011algebra}
\end{definition}

\begin{definition}
	\textbf{Field}: A division ring $D$ is called a field, if the
	multiplication of its elements is always commutative. Thus a field has
	the additional property that $xy = yx$ for all elements $x,y \in D$.
\end{definition}

\bibliographystyle{plain}
\bibliography{../reference_lib/reference}


\end{document}
